\section{About \Voreen} 
\label{section:about}

\subsection{About Voreen}

%\begin{center}
 \includegraphics[scale=0.8,keepaspectratio=true]{./images/voreen-banner.png}
 % voreen-banner.png: 308x83 pixel, 72dpi, 10.86x2.93 cm, bb=0 0 308 83
%\end{center}

Voreen (Volume Rendering Engine) is an open source rapid application development framework for the interactive visualization and analysis of 
multi-modal volumetric data sets. It provides GPU-based volume rendering and data analysis techniques and offers high 
flexibility when developing new analysis workflows in collaboration with domain experts. 
The Voreen framework consists of a multi-platform C++ library, which can be easily integrated into existing applications, 
and a Qt-based stand-alone application. It is licensed under the terms of the \emph{GNU General Public License}. 
The Voreen project has been initiated and is maintained by the \emph{Visualization \& Computer Graphics Research Group} at 
the \emph{University of Münster} as part of the collaborative research center \emph{SFB 656 MoBil (Project Z1, Project Ö)}.
For more information please refer to \url{http://voreen.uni-muenster.de}.

\subsection{About \Voreen}

\includegraphics[scale=0.3,keepaspectratio=true]{./images/logo_bio.png}

\Voreen is an application build on top of the Voreen framework. It features support for large data sets, especially
occuring in the filed of ultramicroscopy, and provides a user interface for the actual application domain,
in contrast to the rapid prototyping environment of Voreen.

Please note that \Voreen is still in beta state. 

\subsection{About this Documentation}

This document provides an introduction to the \Voreen application. It is intended to help users
understand the basic interface and functionality of \Voreen and to provide a starting point for working with the application.
It is, however, by no means a complete reference. If you have any questions regarding \Voreen, please feel free 
to contact us under the following email addresses:

\begin{itemize}
 \item \url{scherzinger@wwu.de}
 \item \url{t.brix@wwu.de}
\end{itemize}

\newpage 


